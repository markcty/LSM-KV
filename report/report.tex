\documentclass{ctexart}

\title{LSM Tree实验报告}
\author{陈天予 519021910045}
\date{\today}
\usepackage{natbib}
\usepackage{graphicx}
\usepackage{enumitem}

\begin{document}

\maketitle

\section{背景介绍}
LSM Tree(Log-structure Merge Tree)数据结构,于1996年在Patrick O’Neil 等人的一篇论文提出。其通过SS-Table的多层储存结构,利用磁盘顺序读写的高效性,实现了性能极高的写操作。LSM Tree被广泛地在各种NoSQL中使用,比如HBase,LevelDB等。

\section{挑战}
\begin{enumerate}
  \item 持久化:之前写过的程序,数据结构都在内存中,很少涉及到文件的读写。LSM Tree通过客制化结构的SSTable实现持久化,需要通过二进制的方式来读写sst文件,因不熟悉c++相关的库函数而导致的bug就不少。
  \item sst文件的debug:由于sst文件在硬盘中,debug的过程中无法实时看到其中的数据,造成许多麻烦和障碍。最终写了一个peekSSTable的小程序扫描sst文件并输出debug信息,解决了debug时的困难。
\end{enumerate}

\section{测试}
\subsection{测试环境}
CPU为apple m1芯片,
\subsection{PUT操作延迟测试}
\subsubsection{测试方法}
随机插入大小在1 Byte到近2 MB大小的字符串,

\end{document}